\documentclass[11pt,a4paper]{article}
\usepackage{amsfonts}
\usepackage{amsmath}
\usepackage{amsthm}
\usepackage[utf8]{inputenc}
\usepackage{hyperref}

\usepackage{minted}
\newminted{py}{%
%		linenos,
		fontsize=\footnotesize,
		tabsize=2,
		mathescape,
}
\newminted{text}{%
%		linenos,
		fontsize=\footnotesize,
		tabsize=2,
		mathescape,
}

\title{Laboratory report 2}
\author{Rodrigo Arias Mallo}

\begin{document}
\maketitle

\section*{Introduction}
\textsl{In this second session we are going to continue with our example of 
assigning tasks to computers in a datacenter.}

\section*{Problem statement}
{\sl

The P2 problem can be formally stated as follows. Given:
\begin{itemize}
\item The set T of tasks. For each task t the amount of resources requested rt 
is specified.
\item The set C of computers. For each computer c the available capacity rc is 
specified.
\end{itemize}
%
Find the assignment of tasks to computers subject to the following constraints:
\begin{itemize}
\item Each task is assigned to exactly one computer.
\item The capacity of each computer cannot be exceeded.
\end{itemize}
%
With the objective to minimize the highest loaded computer. 

}%end of \sl group
%
\section*{Tasks}
\quote{{\sl Note:} The name of each problem is composed of $PXY$, were $X=2$ 
because this is the second lab session, and $Y$ is the letter of the 
correspending exercise.  This naming convention allows me to automatically save 
each run log in a different file.}
\paragraph*{A}
{\sl Implement the P2 model (as problem P2a) in OPL and solve it using CPLEX.  
Use ``boolean'' to define binary variables in OPL.}

As before, I'm going to use pulp and the solver GLPK. Using the previous 
definition of the problem, in report 1, the code is the same. Just a change in 
the variable $x$ from real to binary.

The previous definition
\begin{pycode}
x_tc = LpVariable("x_{}_{}".format(t,c),
	0, 1, LpContinuous)
\end{pycode}
now becomes
\begin{pycode}
x_tc = LpVariable("x_{}_{}".format(t,c),
	0, 1, LpBinary)
\end{pycode}
And the model just runs as before.
\begin{textcode}
% python P2a.py < data.json
Solver status: Optimal
Total load: 1238.47
Total capacity: 1711.13
CPU0 loaded at 0.614
CPU1 loaded at 0.729
CPU2 loaded at 0.799
\end{textcode}
%
\paragraph*{B}
{\sl Compare P1 and P2a in terms of the value of the optimal solution, solving 
algorithm, solving time, and number of variables and constraints.}

The two different logs are stored in {\sl P1-pulp.log} and {\sl P2a-pulp.log}.

In the problem P1, only real numbers were used, it is a pure linear problem.  So 
the solver used by default the simplex algorithm, which has first a stage of 
presolving where some preprocessing was done (but with no change).
%
\begin{textcode}
10 rows, 13 columns, 39 non-zeros
Preprocessing...
10 rows, 13 columns, 39 non-zeros
\end{textcode}
%
Then a scaling process tries to apply a linear transformation to the constraints 
in order to improve its stability. Finally, the simplex algorithm runs with the 
primal variant.
%
\begin{textcode}
Constructing initial basis...
Size of triangular part is 10
      0: obj =   0.000000000e+00 inf =   6.224e+00 (3)
      5: obj =   7.237731791e-01 inf =   0.000e+00 (0)
OPTIMAL LP SOLUTION FOUND
\end{textcode}
%
The first number in each row tell us the number of iterations, so in 5 
iterations an optimal solution is found.

In the problem P2a, there are both integer (binary) variables and real ones. So 
it is a mixed integer (linear) program (MIP). The solver decides to use first 
the simplex algorithm treating all variables as real ones.
%
\begin{textcode}
Solving LP relaxation...
GLPK Simplex Optimizer, v4.60
8 rows, 8 columns, 23 non-zeros
      0: obj =   7.992390778e-01 inf =   1.771e+00 (2)
      1: obj =   7.992390778e-01 inf =   4.441e-16 (0)
OPTIMAL LP SOLUTION FOUND
\end{textcode}
%
Then switches to the integer solver and continues from the result of the linear 
solver. The algorithm used is set by default as branch and cut using heuristic 
by Driebeck and Tomlin.
%
\begin{textcode}
Integer optimization begins...
+     1: mip =     not found yet >=              -inf        (1; 0)
+     1: >>>>>   7.992390778e-01 >=   7.992390778e-01   0.0% (1; 0)
+     1: mip =   7.992390778e-01 >=     tree is empty   0.0% (0; 1)
INTEGER OPTIMAL SOLUTION FOUND
\end{textcode}
%
The time is in both cases less than 50 ms, so its rounded to 0.0s.
\begin{textcode}
Time used:   0.0 secs
\end{textcode}
And the memory is a bit more in the second problem than in the first.
\begin{textcode}
Memory used: 0.0 Mb (39701 bytes)
Memory used: 0.1 Mb (52816 bytes)
\end{textcode}
The number of constraints is in both cases 10, as can be seen in the formulation 
of the problem in {\sl P1-pulp.lp} and {\sl P2a-pulp.lp}. Pulp insert a {\tt 
\_Cxx} tag into each contraint, with {\tt x} being a number. So, searching for 
the contraints lines and counting them gives.
\begin{textcode}
% cat P1-pulp.lp | grep '_C[0-9]*' | wc -l
10
% cat P2a-pulp.lp | grep '_C[0-9]*' | wc -l
10
\end{textcode}
A result of 10 constraints as expected. However, after the preprocessing step, 
in the second problem the number of constraints needed were reduced to 8 (8 
rows).
%
\begin{textcode}
10 rows, 13 columns, 39 non-zeros
12 integer variables, all of which are binary
Preprocessing...
6 constraint coefficient(s) were reduced
8 rows, 8 columns, 23 non-zeros
7 integer variables, all of which are binary
\end{textcode}
%
The number of variables is at the start of both problems 13 (13 columns).
\begin{textcode}
10 rows, 13 columns, 39 non-zeros
\end{textcode}
But after the preprocessing step, the second problem ends with only 8 columns, 
so just 8 variables, 7 of them are binary.

The solution has the same optimum value of the objective function, with both 
resources used of $1238.47$. However the distribution of tasks is not the same 
anymore.
%
For the P1 problem, the distribution of tasks is as follows
\begin{textcode}
% python P1.py < data.json
...
100.00% of task 0 runs on computer 2
 65.25% of task 1 runs on computer 0
 34.75% of task 1 runs on computer 2
 83.33% of task 2 runs on computer 1
 16.67% of task 2 runs on computer 2
100.00% of task 3 runs on computer 1
\end{textcode}
%
But in the P2a
\begin{textcode}
% python P2a.py < data.json
...
100.00% of task 0 runs on computer 1
100.00% of task 1 runs on computer 2
100.00% of task 2 runs on computer 0
100.00% of task 3 runs on computer 1
\end{textcode}


%
\paragraph*{C}
{\sl Solve a new problem P2c with the following data file, where a new task has 
been added. Analyze the obtained results.}
%
\begin{textcode}
nTasks=5;
nCPUs=3;

rt=[261.27 560.89 310.51 105.80 344.7];
rc=[505.67 503.68 701.78];
\end{textcode}
%
First, we need the data to be in {\sl json} format to be able to read it with 
python.
\begin{textcode}
% cat data-p2.json
{
        "rc": [505.67, 503.68, 701.78],
        "rt": [261.27, 560.89, 310.51, 105.8, 344.7]
}
\end{textcode}
%
Then execute the program with the new data file.
%
\begin{textcode}
% python P2c.py < data-p2.json
Solver status: Infeasible
Total load: 1583.17
Total capacity: 1711.13
CPU0 loaded at 0.000
CPU1 loaded at 0.000
CPU2 loaded at 0.000
\end{textcode}
%
It seems now that the problem becomes unfeasible. The total resources needed by 
all the tasks are $1583.17$ and the total resources available are $1711.13$.  
However there is no distribution of tasks in the computers, without splitting 
any task in different computers, that can be found. So there is no solution for 
this problem.
%

\paragraph*{D}
{\sl Modify the P2 model to allow rejecting tasks, i.e. some tasks might not be 
processed. To this aim, consider a new parameter K defining the maximum number 
of tasks that can be rejected (P2d). Analyze the obtained results varying the 
value of K.}

First the model was previously defined, with $z$
\begin{equation}
  z = \max_j L_j
  = \max_j \left\{ \frac{1}{RC_j} \sum_{i}{RT_i \cdot x_{ij}} \right\}
\end{equation}
And the model
\begin{align}
\text{Minimize}   \quad & z\\
\text{Subject to} \quad & \sum_j x_{ij} = 1 \label{eq:c1_old}\\
& \sum_i RT_i \cdot x_{ij} \le RC_j \\
& z \ge \frac{1}{RC_j} \sum_{j}{RT_i \cdot x_{ij}}
\end{align}
With $1 \le i \le N$ and $1 \le j \le M$.

Let $Y = \{y_1, y_2,\ldots,y_n \}$ be a set of $N$ new $y_i$ binary variables, 
with $N$ being the number of tasks and $1 \le i \le N$. If a task $t_i$, is 
accepted, $y_i = 1$, otherwise it is rejected and $y_i = 0$. 

Then the number of rejected variables becomes $R = N - \sum_i y_i$. To add the 
restriction of the maximum number of rejected variables, a new constraint can be 
formulated.
\begin{align}
%\setcounter{equation}{5}
R &\le K \nonumber \\
N - \sum_i y_i &\le K \nonumber \\
\sum_i y_i &\ge N - K
\end{align}
%
Also the equation~\eqref{eq:c1_old} now should be changed, because some tasks 
can be rejected, making the sum $\sum_j x_{ij}$ equal to 0 for those tasks 
rejected.  So, as $y_i = 0$ when the task $t_i$ is rejected, and 1 otherwise, 
can be rewritten as
\begin{equation}
\sum_j x_{ij} = y_i
\end{equation}
Giving the new model
\begin{align}
\text{Minimize}   \quad & z\\
\text{Subject to} \quad & \sum_j x_{ij} = y_i \\
& \sum_i RT_i \cdot x_{ij} \le RC_j \\
& z \ge \frac{1}{RC_j} \sum_{j}{RT_i \cdot x_{ij}} \le z\\
& \sum_i y_i \le N - K
\end{align}
With $1 \le i \le N$ and $1 \le j \le M$.

The model in the python program {\sl P2d.py} follows the formal description
\begin{pycode}
# The problem
prob = LpProblem(problem, LpMinimize)

# Constraint 1
for t in T:
	prob += lpSum([x[t][c] for c in C]) == y[t]

# Constraint 2
for c in C:
	prob += lpSum([rt[t] * x[t][c] for t in T]) <= rc[c]

# Constraint 3
for c in C:
	prob += z >= (1.0 / rc[c]) * lpSum([rt[t] * x[t][c] for t in T])

# Contraint 4
prob += lpSum([y[t] for t in T]) >= N - K


# Objective function
prob += z
\end{pycode}
%
The result of the execution, with $0 \le K \le 3$.
\begin{textcode}
% for K in 0 1 2 3; do echo "K=$K"; python P2d.py $K < data-p2.json; done
K=0
Solver status: Infeasible
Resources used were 0.00 of 1711.13 available
There were 5 of 5 tasks rejected with a total of 1583.17 resources
A total of 1711.13 resources were unused. K=0
Task 0 with 261.27 of resources was rejected
Task 1 with 560.89 of resources was rejected
Task 2 with 310.51 of resources was rejected
Task 3 with 105.80 of resources was rejected
Task 4 with 344.70 of resources was rejected
K=1
Solver status: Optimal
Resources used were 1022.28 of 1711.13 available
There were 1 of 5 tasks rejected with a total of 560.89 resources
A total of 688.85 resources were unused. K=1
Task 0 with 261.27 of resources runs on computer 1
Task 1 with 560.89 of resources was rejected
Task 2 with 310.51 of resources runs on computer 0
Task 3 with 105.8 of resources runs on computer 2
Task 4 with 344.7 of resources runs on computer 2
K=2
Solver status: Optimal
Resources used were 677.58 of 1711.13 available
There were 2 of 5 tasks rejected with a total of 905.59 resources
A total of 1033.55 resources were unused. K=2
Task 0 with 261.27 of resources runs on computer 0
Task 1 with 560.89 of resources was rejected
Task 2 with 310.51 of resources runs on computer 2
Task 3 with 105.8 of resources runs on computer 1
Task 4 with 344.70 of resources was rejected
K=3
Solver status: Optimal
Resources used were 367.07 of 1711.13 available
There were 3 of 5 tasks rejected with a total of 1216.10 resources
A total of 1344.06 resources were unused. K=3
Task 0 with 261.27 of resources runs on computer 2
Task 1 with 560.89 of resources was rejected
Task 2 with 310.51 of resources was rejected
Task 3 with 105.8 of resources runs on computer 1
Task 4 with 344.70 of resources was rejected
\end{textcode}
%
As a result of increasing $K$ the number of rejected tasks also increases. The 
objective function is defined to minimize the load of the highest load computer.  
So allowing more rejected tasks, the model just keeps the minimum number of task 
in the computers, minimizing the load. It can be observed that the task with the 
greatest resource demand is the next rejected when increasing $K$.

\paragraph*{E}
{\sl Modify the objective function so as to minimize the amount of not served 
load (P2e)}

The not served load, can be described in terms of the unused load of all the 
computers, $U$. Let $U_j$ be the unused load of the computer $c_j$.  Then
%
$$ U_j = \frac{RC_j - \sum_i RT_i \cdot x_{ij}}{RC_j} =
1 - \frac{1}{RC_j} \sum_i RT_i \cdot x_{ij}$$
%
And as $U = \sum_j U_j$
%
$$U = \sum_j \left( 1 - \frac{1}{RC_j} \sum_i RT_i \cdot x_{ij} \right) =
M - \sum_j \left( \frac{1}{RC_j} \sum_i RT_i \cdot x_{ij} \right)
$$
%
With this definition of $U$, the model can now be written without $z$, modifying 
the previous constraint
%
\begin{align}
\text{Minimize}   \quad & U\\
\text{Subject to} \quad & \sum_j x_{ij} = y_i \\
& \sum_i RT_i \cdot x_{ij} \le RC_j \\
%& U = \sum_j \left( RC_j -  \sum_i RT_i \cdot x_{ij} \right)\\
& U = M - \sum_j \left( \frac{1}{RC_j} \sum_i RT_i \cdot x_{ij} \right) \\
& \sum_i y_i \le N - K
\end{align}
With $1 \le i \le N$ and $1 \le j \le M$.

The new model now becomes
%
\begin{pycode}
# The problem
prob = LpProblem(problem, LpMinimize)

# Constraint 1
for t in T:
	prob += lpSum([x[t][c] for c in C]) == y[t]

# Constraint 2
for c in C:
	prob += lpSum([rt[t] * x[t][c] for t in T]) <= rc[c]

# Constraint 3
prob += U == M - lpSum([1/rc[c] * lpSum(
	[rt[t] * x[t][c] for t in T]) for c in C])

# Constraint 4
prob += lpSum([y[t] for t in T]) >= N - K

# Objective function
prob += U
\end{pycode}
%
Using any $K>0$, the result of the execution is always the same
%
\begin{textcode}
% python P2e.py 1 < data-p2.json
Solver status: Optimal
Resources used were 1321.90 of 1711.13 available
There were 1 of 5 tasks rejected with a total of 261.27 resources
A total of 389.23 resources were unused. K=1
Task 0 with 261.27 of resources was rejected
Task 1 with 560.89 of resources runs on computer 2
Task 2 with 310.51 of resources runs on computer 0
Task 3 with 105.8 of resources runs on computer 1
Task 4 with 344.7 of resources runs on computer 1
\end{textcode}
%
The rejected task is now the task 0 instead of the task 1. The load of the 
computers is now
\begin{textcode}
Computer 0 loaded at 61.41%
Computer 1 loaded at 89.44%
Computer 2 loaded at 79.92%
\end{textcode}
Compared with the previous problem P2d, with $K=1$
\begin{textcode}
Computer 0 loaded at 61.41%
Computer 1 loaded at 51.87%
Computer 2 loaded at 64.19%
\end{textcode}
There is now a better load usage.

\paragraph*{F}
{\sl Compare all three models (P2a, P2d, P2e) in terms of number of variables, 
constraints and execution time.}

Asuming that the three models to compare are P2a, P2d and P2e. The number of 
variables and constraints are shown in parenthesis after the preprocessing step.

\begin{tabular}{l l l l l}
Model & Variables & Constraints & Time (s) & Memory (bytes)\\
P2a & 13 (8) & 10 (8) & 0.0 & 52816\\
P2d & 21 (19) & 12 (12) & 0.0 & 61859\\
P2e & 21 (18) & 10 (10) & 0.0 & 58866\\
\end{tabular}


\end{document}
