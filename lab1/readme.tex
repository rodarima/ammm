\documentclass[11pt,a4paper,twocolumn]{article}
\usepackage{amsfonts}
\usepackage{amsmath}
\usepackage{amsthm}
\usepackage[utf8]{inputenc}
\usepackage{hyperref}

\usepackage{minted}
\newminted{py}{%
%		linenos,
		fontsize=\footnotesize,
		tabsize=2,
		mathescape,
}

\title{Laboratory report 1}
\author{Rodrigo Arias Mallo}

\begin{document}
\maketitle

\section*{Introduction}
\textsl{The problem is about assigning a set of tasks (processes) to a set of
computers in a data center.}

So we have some set of tasks $T = \{t_1, t_2, \ldots, t_N\}$, and some set of
computers $C = \{c_1, c_2, \ldots, c_M\}$.

In addition, each tast $t_i$ requests some amount of processing resources. Let
$RT_i$ be the amount of resources need by the task $i$.

Also, each computer $c_j$ has \textbf{one} CPU, with a processing capacity. Let
$RC_j$ be the processing capacity of the computer $j$.

\textsl{Since the power consumption of the computers increases heavily with the
load, the objective consists in distributing the tasks among all the computers
so to minimize the load of the computer running at the highest load.}

Let $L_j$ be the load and $P_j$ be the power consumption of the computer $c_j$.
Then, there is an asumption that if $L_j = \alpha$ and $P_j = \beta$, then for a
positive $\epsilon$, if the load is now $L_j = \alpha + \epsilon$ implies more power
consumption, $P_j > \beta$.

Also, the load of a computer, means that the processor does real work by a
fraction of the amount of work capable of doing, so $L_j \le 1$

So if $L_{max} = \max_{j} L_j$ is the highest load, we should reduce it. That
could be expressed as an objective function $\min(z) = \min(L_{max})$.

But be aware, could be a misconception here. Suposse two computers $c_1$ and 
$c_2$, both with the same processing resources $RC_1 = RC_2$. If the computer 
$c_1$ consumes linearly with the load, say $P_1 = K \cdot L_1$, and another 
computer, less eficient, $c_2$ consumes five times more with the load, $P_2 = 5K 
\cdot L_2$.  Then, reducing the load of the highest loaded computer to the 
minimum does not always reduce the overall power consumption to the minimum.  
Which could be useful.

Let $L_1 = 1/2$ and $L_2 = 1/2$, then all the power computer power consumption 
is $\sum_j P_j = K \cdot L_1 + 5K \cdot L_2 = 1/2 K + 5/2K = 3K$. And the 
highest load is at the minimum (1/2) for both computers. But, using only the 
eficient computer at $L_1 = 1$ and $L_2 = 0$, only consumes $\sum_j P_j = K 
\cdot L_1 = K$, three times less.

% TODO: Measure diference.
Reducing the highest load does not imply reducing the power consumption.

\subsection*{Problem statement}

\sl
The problem can be formally stated as follows. Given:
\begin{itemize}
\item The set $T$ of tasks. For each task $t_i$ the amount of processing 
resources requested $RT_i$ is specified.
\item The set $C$ of computers. For each computer $c_j$ the available processing 
capacity $RC_j$ is specified.
\end{itemize}
Find the assignment of tasks to computers subject to the following constraints: 
\begin{itemize}
\item Each task is assigned to some computers so that the entire requested 
processing load is served. Note that this means that a single task can consume 
processing capacity of several computers.
\item The processing capacity of each computer cannot be exceeded.  with the 
objective to minimize the load of the highest loaded computer.
\end{itemize}

\subsection*{LP formulation}
The P1 problem can be modeled as a Linear Program. To this end, the following 
sets and parameters are defined:
%
\begin{center}
\begin{tabular}{l l}
$T$    & Set of $t_i$ tasks. \\
$C$    & Set of $c_j$ computers.\\
$RT_i$ & Resources requested by task $t_i$.\\
$RC_j$ & Available capacity of computer $c_j$.
\end{tabular}
\end{center}
%
The following decision variables are also defined:
\begin{center}
\begin{tabular}{p{.1\linewidth} p{.8\linewidth}}
$x_{ij}$ & Positive real with the percentage of resources requested by task 
$t_i$ that are served from computer $c_j$.\\
%
$z$ & Positive real with the load of the highest loaded computer.
\end{tabular}
\end{center}
%
The load $L_j$ of the computer $c_j$ can be expressed in terms of the sum of 
each contribution from the task $t_i$. The contribution is computed as the 
resource requested by the task $t_i$ times the fraction running on computer 
$c_j$, being $RT_i \cdot x_{ij}$. So the load $L_j$ is just the sum of all 
resources requested over the total resources of the computer $c_j$, $L_j = 
\frac{1}{RC_j} \sum_i RT_i \cdot x_{ij}$.

Note that the variable z stores the load of the highest loaded computer, that 
is:
%
\begin{equation}
  z = \max_j L_j
  = \max_j \left\{ \frac{1}{RC_j} \sum_{i}{RT_i \cdot x_{ij}} \right\}
\end{equation}
%
Finally, the LP model for the P1 problem is as follows:

\noindent
\begin{minipage}{\columnwidth}
\begin{align}
\text{Minimize } & z\\
\text{Subject to}& \nonumber \\
& \sum_j x_{ij} = 1 \\
& \sum_i RT_i \cdot x_{ij} \le RC_j \\
& z \ge \frac{1}{RC_j} \sum_{j}{RT_i \cdot x_{ij}}
\end{align}
With $1 \le i \le N$ and $1 \le j \le M$.
%\end{equation}
\end{minipage}
\rm

\section*{Tasks}
\paragraph*{A}
\textsl{Explain the P1 model in eq. 2 to 5. Specifically, define each of the
constraints.}

The eq.~2 just states that the load of the maximum loaded computer should be 
minimized.

The eq.~3 ensures that for each task $t_i$, all the portion of the task is fully 
distributed on the computers. So the sum of each fraction of resources for all 
the computers for the task $t_j$, should be 1, to complete the task.

The eq.~4 prevents for a computer $c_j$, being requested more resources than it 
can process, $RC_j$. So the sum of all the fraction of resources requested by 
the tasks, should not be greater than the capacity of the computer $c_j$.

Finally, the last eq. 5, is a tricky one. The $z$ variable is being minimized, 
so forces the right hand side to be smaller. And that corresponds to the load of 
the computer $c_j$ as said before. So the $z$ acts as a barrier trying to reduce 
the load the computer $c_j$. When a computer with load $L_j$ has the maximum 
load $L_{max}$, then all the other inequations automatically holds. So the 
solver will try to reduce the load of tho computer $c_j$, so reducing always the 
maximum load $L_{max}$.

\paragraph*{B}
\textsl{Explain how equation 1 is implemented in the model P1. Specifically, why 
z is guaranteed to be equal to the load of the highest loaded computer?}

The eq. 1 just computes the load of the computer $c_j$, called $L_j$ based on 
the sum of all fraction of resources from all the tasks, assigned to the 
computer $c_j$. And divides the sum over the total computing resource available, 
to get the load.

Then, from all the loads $L_j$, the maximum load is selected, $L_{max}$. The 
selection of the maximun load is implemented as a constraint, in the eq. 5. The 
solver tries to minimize $z$, and the maximum load is the first which resists 
the reduction of $z$.

The value of $z$ will be the maximum load $L_{max}$, because if it were greater, 
then it could be minimized until the value of $L_{max}$. And cannot be less, 
because the eq. 5.

\paragraph*{C}
\textsl{Implement the P1 model in OPL following the steps in section 3 and solve 
it using CPLEX.}

As I don't have the software in my own machine, I will use another method. I 
will be using the GLPK toolkit as a solver, and implement the problem in python 
using pulp. In this way, another solver could be choosen in other workstations, 
using CPLEX when available.

The semantics are quite similar, so it shall be easy to follow. The model 
proposed should have the data and the model of the problem in different files.  
To load the data, just two lines are enough.
%
\begin{pycode}
rt = [261.27, 560.89, 310.51, 105.80]
rc = [505.67, 503.68, 701.78]
\end{pycode}
%
Then the ranges are computed, based on the length of $RT$ and $RC$
%
\begin{pycode}
T = range(len(rt))
C = range(len(rc))
\end{pycode}
%
For the other part of the problem, the variables \texttt{x} and \texttt{z} are 
created first, corresponding with $x_{ij}$ and $z$.
%
\begin{pycode}
x = []
for t in T:
	x_t = []
	for c in C:
		x_name = "x_{}_{}".format(t,c)
		x_tc = LpVariable(
			x_name,
			0,
			1,
			LpContinuous)
		x_t.append(x_tc)
	x.append(x_t)

z = LpVariable("z", 0, 1, LpContinuous)
\end{pycode}
%
To continue with, the contraints.

\begin{pycode}
# The problem
prob = LpProblem("problem", LpMinimize)

# Constraint 1
for t in T:
	prob +=
		lpSum(
			[x[t][c] for c in C]) == 1.00

# Constraint 2
for c in C:
	prob +=
		lpSum(
			[rt[t] * x[t][c] for t in T]) <= rc[c]

# Constraint 3
for c in C:
	prob +=
		z >= (1.0 / rc[c]) * lpSum(
			[rt[t] * x[t][c] for t in T])

# Objective function
prob += z
\end{pycode}
%
And now, just solve the problem, using the solver of choice, in this case is 
GLPK.
\begin{pycode}
status = prob.solve(GLPK(msg=1))
print("Solver status: {}".format(
	LpStatus[status]))
\end{pycode}
%
Finally, print the total load.
%
\begin{pycode}
total_load = sum(rt)
print("Total load: {}".format(total_load))
\end{pycode}
And the load for each computer.
\begin{pycode}
for c in C:
	load = 0
	for t in T:
		load +=
			rt[t] * value(x[t][c])
	load = (1/rc[c]) * load
	print("CPU{} loaded at {:.3f}"
		.format(c, load))
\end{pycode}
After running the program \texttt{lab1-a.py}, an optimal solution is reached.
\begin{pycode}
python lab1-a.py
Solver status: Optimal
Total load: 1238.47
Total capacity: 1711.13
CPU0 loaded at 0.724
CPU1 loaded at 0.724
CPU2 loaded at 0.724
\end{pycode}

\paragraph*{D}
\textsl{Write a processing block to check whether the load requested by all the 
tasks can be served by the computers.}

The resources needed for the tasks are the sum of each, $RT_{all} = \sum_i 
RT_i$.  And for the available resources, $RC_{all} = \sum_i RC_i$. If $RT_{all} 
> RC_{all}$, then there cannot be distributed with all loads $L_j \le 1$. So 
checking for enough resources is done as follows.
\begin{pycode}
if sum(rt) > sum(rc):
	print("Not enough computer resources")
	exit(1)
\end{pycode}

\paragraph*{E}
\textsl{Implement the P1 model in OPL following the steps in section 4 and solve 
it using CPLEX.}

The key here is to be able to modify the data, without dealing with the model.  
So now, instead of write the data in the program, it will be loaded from a 
external source, a json file. As an example, the provided data.
\begin{pycode}
{
	"rc": [505.67, 503.68, 701.78],
	"rt": [261.27, 560.89, 310.51, 105.8]
}
\end{pycode}
%
Now, to load the data from a file, just adding some lines instead of the data, 
gives.
%
\begin{pycode}
data = json.load(sys.stdin)
rc = data['rc']
rt = data['rt']
\end{pycode}
%
And it can be tested with different data files.
%
\begin{pycode}
python lab1-b.py < data.json
Solver status: Optimal
Total load: 1238.47
Total capacity: 1711.13
CPU0 loaded at 0.724
CPU1 loaded at 0.724
CPU2 loaded at 0.724

python lab1-b.py < data-bad.json
Not enough computer resources
\end{pycode}
%
Even with all the data files. 
%
\begin{pycode}
find -name "data*.json" -print \
	-exec sh -c "python lab1-b.py < {}" \;

./data-bad.json
Not enough computer resources
./data.json
Solver status: Optimal
Total load: 1238.47
Total capacity: 1711.13
CPU0 loaded at 0.724
CPU1 loaded at 0.724
CPU2 loaded at 0.724
...
\end{pycode}

\paragraph*{F}
\textsl{Analyze the computer's processing capacity and indicate whether is 
possible to reduce it.}

The total resource capacity of all the computers, is 1711.13, which is greater 
than the required by the tasks, 1238.47. So instead of using three computers 
with capacities [505.67, 503.68, 701.78], two computers of 701.78 could be used.  
Or another computer with less capacity [505.67, 503.68, 300.00], giving

\begin{pycode}
python lab1-b.py < data-less.json 
Solver status: Optimal
Total load: 1238.47
Total capacity: 1309.35
CPU0 loaded at 0.946
CPU1 loaded at 0.946
CPU2 loaded at 0.946
\end{pycode}

\section*{Requirements}
I have ran all the examples in my own machine, running an UNIX system. The 
version of python used was 3.5.2. The version of pulp was 1.6.1. And of GLPK was 
4.60.

More info of how to install pulp and GLPK can be found in 
\url{https://pythonhosted.org/PuLP/} and 
\url{https://www.gnu.org/software/glpk/}

\end{document}

