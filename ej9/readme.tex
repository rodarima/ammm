\documentclass[11pt,a4paper]{article}
\usepackage{amsfonts}
\usepackage{amsmath}
\usepackage{amsthm}
\usepackage[utf8]{inputenc}
\usepackage{hyperref}

\usepackage{minted}
\newminted{py}{%
%		linenos,
		fontsize=\footnotesize,
		tabsize=2,
		mathescape,
}
\newminted{text}{%
%		linenos,
		fontsize=\footnotesize,
		tabsize=2,
		mathescape,
}

\title{Exercise 9}
\author{Rodrigo Arias Mallo}

\begin{document}
\maketitle

\section*{Factory}

The factory produces $N=4$ different products, $P = \{p_1, p_2, p_3, p_4 \}$.  
The quantity $q_{ij}$ measures the units of products $p_i$ that were produced at 
the end of the day $j$. Also, $w_{ij}$ is 1 if any $p_i$ were produced, or 0 
otherwise. If there is any production of $p_i$ on day $j$, both $w_{ij}$ and 
$q_{ij}$ are non-zero. As $w_{ij}$ is binary,
%
$$ q_{ij} \ge w_{ij} $$
%
Also when there is some production $q_{ij}$, $w_{ij}$ should be 1,
%
$$ BIG \cdot w_{ij} \ge q_{ij} $$
%
Being $BIG$ a big number, greater than any posible quantity $q_{ij}$. Each day 
$j$, the factory works a number of hours $h_{ij}$ in the product $p_i$.  The 
factory is never idle, it is always working on a product. The rate $r_i$ at 
which the product $p_i$ is produced is different for each product, but constant 
for each day. Also, the quantity
%
$$ q_{ij} = h_{ij} \cdot r_i $$
%
Only one product is produced each day, so $ \sum_i w_{ij} = 1 $. The number of 
hours for the current working product $p_i$ at the day $j$ is $h_{ij} = 
24 \cdot w_{ij}$, except for $p_2$. In this case, if $p_1$ was produced the past 
	 day, 5 hours should be removed, $h_{2j} = w_{2j} \cdot (24 - w_{1 j-1} \cdot 
	 5)$.

However, the multiplication of two variables is not allowed in linear programs. 
To solve this problem, a new binary variable $pp_j$ is 1 if at day $j$ the product 
$p_2$ is produced, and at day $j-1$ was $p_1$. If $w_{1j-1}$ or $w_{2j}$ are 0, force 
$pp_j$ to be also 0,
%
$$ pp_j \le w_{1j-1}, \quad pp_j \le w_{2j} $$
%
But if they are both 1, compute the sum minus one, and force $pp_j$ to be 
greater or equal, $ pp_j \ge w_{1j-1} + w_{2j} - 1 $. Now the problem can be 
solved using $pp_j$, $$h_{2j} = w_{2j} \cdot 24 - pp_j \cdot 5$$
%
Each day $j$ a number or units of $p_i$ are demanded, $d_{ij}$, which are fixed.  
The other units not demanded, are stored in stock. The number of products $p_i$ 
in stock at the end of the day $j$, $s_{ij}$ is at the beginning $s_{i0} = 
cs_i$.  The current stock $cs_i$ of the product $p_i$ is the number of items at 
the beginning of the problem, at the starting of the day 1, or at the end of day 
0. The stock at day $j$ is the previous one, plus the produced units, minus the 
	 demanded, $s_{ij} = s_{ij-1} + q_{ij} - d_{ij}$. And at the end of the week, 
	 at $j=7$, the stock should be at least 1750 for any product, $s_{i7} \ge 
	 1750$.

Holding the product $p_i$ in stock costs $H_i$ per day and product. The cost of 
holding a product $p_i$ in stock at day $j$ is $c_{ij} = H_i \cdot s_{ij}$.
%
Finally, the total cost $C$, which is the objective fuction, is the sum of all 
the holding costs,
$$ C= \sum_i \sum_j c_{ij} $$
%

\subsection*{Implementation}

To implement the problem in pulp, first the parameters $r_i$, $d_{ij}$, $H_i$ 
and $cs_i$ are defined.
%
\begin{pycode}
# Rate of production
r = [200, 450, 390, 250]

# Demanded units
d=[[0, 1500, 1700, 1900, 1000, 2000 , 500,  500],
	 [0, 4000,  500, 1000, 3000,  500, 1000, 2000],
	 [0, 2000, 2000, 3000, 2000, 2000, 2000,  500],
	 [0, 3000, 2000, 2000, 1000, 1000,  500,  500]]

# Cost of holding per day and unit
H = [1.5, 1.5, 2.5, 2.5]

# Current stock
cs = [5000, 7000, 9000, 8000]

PRODUCTS = 4
DAYS = 7
\end{pycode}
%
Then the variables $w_{ij}$, $q_{ij}$, $h_{ij}$, $s_{ij}$, $c_{ij}$, $pp_j$ and 
$C$ are defined.
%
\begin{pycode}
# Working on product i at day j.
w = matrix_var(I, J0, 'w', kind=LpBinary)

# Quantity of products i produced at day j
q = matrix_var(I, J0, 'q', kind=LpInteger)

# Hours of production of product i at day j
h = matrix_var(I, J0, 'h', kind=LpContinuous)

# Stock of product pi at end of day j
s = matrix_var(I, J0, 's', kind=LpInteger)

# Cost of holding product i at day j
c = matrix_var(I, J0, 'c', kind=LpContinuous)

# HACK: First product produced yesterday and second product now?
pp = []
for j in J0:
	pp_j = LpVariable("pp_{}".format(j), 0, None, LpBinary)
	pp.append(pp_j)

# Total cost of stock
C = LpVariable("C", 0, None, LpContinuous)
\end{pycode}
%
Finally, the constraints
%
\begin{pycode}
# The problem
prob = LpProblem(problem, LpMinimize)

# Each day only 1 product is produced
for j in J:
	prob += lpSum([w[i][j] for i in I]) == 1

# pp is 0 if p1 was not produced yesterday
for j in J:
	prob += pp[j] <= w[p1][j-1]

# pp is 0 if p2 is not produced today
for j in J:
	prob += pp[j] <= w[p2][j]

# pp is 1 if both p1 yesterday and p2 today
for j in J:
	prob += pp[j] >= w[p1][j-1] + w[p2][j] - 1


# At most 24h / day of production
for i in I:
	if i == p2:
		# 5 h less for product p2 after p1, HACK
		for j in J:
			prob += h[p2][j] == w[i][j] * 24 - pp[j] * 5
	else:
		for j in J:
			prob += h[i][j] == w[i][j] * 24

# No work done on day 0
for i in I:
	prob += h[i][0] == 0

# If product i is not produced (q=0), then we are not working on it (w=0)
for i in I:
	for j in J:
		prob += q[i][j] >= w[i][j]

# If working on it (w=1), then it is produced (q>0)
for i in I:
	for j in J:
		prob += q[i][j] <= w[i][j] * BIGNUM

# The quantity is hours per rate
for i in I:
	for j in J:
		prob += q[i][j] == h[i][j] * r[i]

# No quantity produced on day 0
for i in I:
	prob += q[i][0] == 0

# The current stock at day 0
for i in I:
	prob += s[i][0] == cs[i]

# All stocks at the end of the week >= 1750
for i in I:
	prob += s[i][7] >= 1750

# The stock is the previous day + produced - demanded
for i in I:
	for j in J:
		prob += s[i][j] == s[i][j-1] + q[i][j] - d[i][j]

# Cost of holding product i at end of day j
for i in I:
	for j in J:
		prob += c[i][j] == H[i] * s[i][j]

# Cost at day 0 is zero
for i in I:
	prob += c[i][0] == 0

# The last week produced product is 3
prob += w[p3][0] == 1

# Total cost of holding the products
C = lpSum([ lpSum([c[i][j] for j in J])  for i in I])

# Objective function
prob += C
\end{pycode}
%
\subsection*{Results}
%
After solving the problem, an optimal solution was found. The plan for the week 
is described in the execution of the program.

\begin{textcode}
% python 9.py
 
Solver status: Optimal
C = 312725.0
---------------------------------------------------------------------
Day 0. Working for 0.0 hours in product 3
Demanded:          0,          0,          0,          0
Produced:          0,          0,          0,          0
Stocks:         5000,       7000,       9000,       8000
Holding:         0.0,        0.0,        0.0,        0.0
---------------------------------------------------------------------
Day 1. Working for 24.0 hours in product 1
Demanded:       1500,       4000,       2000,       3000
Produced:       4800,          0,          0,          0
Stocks:         8300,       3000,       7000,       5000
Holding:     12450.0,     4500.0,    17500.0,    12500.0
---------------------------------------------------------------------
Day 2. Working for 24.0 hours in product 1
Demanded:       1700,        500,       2000,       2000
Produced:       4800,          0,          0,          0
Stocks:        11400,       2500,       5000,       3000
Holding:     17100.0,     3750.0,    12500.0,     7500.0
---------------------------------------------------------------------
Day 3. Working for 19.0 hours in product 2
Demanded:       1900,       1000,       3000,       2000
Produced:          0,       8550,          0,          0
Stocks:         9500,      10050,       2000,       1000
Holding:     14250.0,    15075.0,     5000.0,     2500.0
---------------------------------------------------------------------
Day 4. Working for 24.0 hours in product 4
Demanded:       1000,       3000,       2000,       1000
Produced:          0,          0,          0,       6000
Stocks:         8500,       7050,          0,       6000
Holding:     12750.0,    10575.0,        0.0,    15000.0
---------------------------------------------------------------------
Day 5. Working for 24.0 hours in product 3
Demanded:       2000,        500,       2000,       1000
Produced:          0,          0,       9360,          0
Stocks:         6500,       6550,       7360,       5000
Holding:      9750.0,     9825.0,    18400.0,    12500.0
---------------------------------------------------------------------
Day 6. Working for 24.0 hours in product 1
Demanded:        500,       1000,       2000,        500
Produced:       4800,          0,          0,          0
Stocks:        10800,       5550,       5360,       4500
Holding:     16200.0,     8325.0,    13400.0,    11250.0
---------------------------------------------------------------------
Day 7. Working for 24.0 hours in product 1
Demanded:        500,       2000,        500,        500
Produced:       4800,          0,          0,          0
Stocks:        15100,       3550,       4860,       4000
Holding:     22650.0,     5325.0,    12150.0,    10000.0
---------------------------------------------------------------------
\end{textcode}
%
The amount is $C = 312725$. However, just allowing the factory to stop producing 
units when needed in the day, or slowing the rate, $C$ could be reduced to 
146525, less than the half.

\end{document}
