\documentclass[11pt,a4paper]{article}
\usepackage{amsfonts}
\usepackage{amsmath}
\usepackage{amsthm}
\usepackage[utf8]{inputenc}
\usepackage{hyperref}

\usepackage{minted}
\newminted{py}{%
%		linenos,
		fontsize=\footnotesize,
		tabsize=2,
		mathescape,
}
\newminted{text}{%
%		linenos,
		fontsize=\footnotesize,
		tabsize=2,
		mathescape,
}

\newcommand*\mat[1]{ \begin{pmatrix} #1 \end{pmatrix}}
\newcommand*\arr[1]{ \begin{bmatrix} #1 \end{bmatrix}}

\title{Exam January 13th 2014}
\author{Rodrigo Arias Mallo}

\begin{document}
\maketitle

\section*{Exercise 1}

{\sl A set of users $U$ need to be connected to the Internet. For that purpose, 
we have available a set of access points $A$ where we could install several 
routers (at most one per access point). We want to find out in which access 
points we should install a router and to which access point each user should be 
connected to, in order to minimize cost and satisfy other constraints that are 
defined in the following.}

\subsection*{A}

\textsl{Let us first assume that we can only install one type of router. The 
cost of installing such a router in an access point has two components: a fixed 
cost $f$, and an additional cost $c$ for each user that is connected to it. Such 
a router has a given capacity $k$ and we know, for each user $u \in U$, the 
amount $cr_u$ of capacity units it consumes from the router it is connected to.}

\textsl{Write down an integer linear program to find out in which access points 
we should install a router and which access point each user should be connected 
to in order to minimize cost and not to exceed the capacity of each router.  
Additionally, create an OPL project to solve this problem. Problem data will be 
found in the file \texttt{examA.dat}. Use the file initialModel.mod as the 
initial model and complete it.  Save this file as \texttt{modelA.mod} and send 
it through ``El Racó''.}

\paragraph*{Model}

An access point is a place where a router can be installed, or can be empty.  
Some quantities like $f$ or $c$ are fixed, but others like the position of each 
router can vary as we evaluate different solutions. To prevent any confusion, 
lets rename the fixed quatities using a greek symbol.
%
$$ \alpha = f,\, \beta = c $$
%
Also the vector $cr$ is renamed to just $R$ to prevent any confusion with the 
matrix product.
%
To determine whether or not a router should be placed at an access point, the 
column vector $P \in \{0,1\}^{1 \times A}$ where $P_a$ is 1 if a router is 
placed at $a$, or 0 otherwise.
%An example of $P$ for 5 access points with 3 routers.
%
%$$ P^T = \mat{1 & 0 & 1 & 1 & 0} $$
%
Lets define a matrix $C \in \{0,1\}^{U \times A}$ so the element $C_{ua}$ is 1 
if the user $u$ is connected to the router at the access point $a$, and 0 
otherwise.
%An example of C for 4 users and 5 access points with 3 routers.
%%
%$$ C = \mat{
%	1 & 0 & 0 & 0 & 0 \\
%	1 & 0 & 0 & 0 & 0 \\
%	0 & 0 & 1 & 0 & 0 \\
%	0 & 0 & 0 & 1 & 0 \\
%} $$
%%
The cost of a router $M_a$ at $a$ can be rewritten as
%
$$ M_a = \alpha P_a + \beta \sum_{u \in U} C_{ua} $$
%
Also a vector with all the costs of the routers in the access points $M \in 
\mathbb{R}^{1 \times A}$, where $M_a$ is the cost of the router at $a$, is 
defined as
%
$$ M = \alpha P + \beta C^T \bf{1} $$
%
%As an example, using the previous matrices,
%
%\begin{equation*}
%\begin{split}
%M=\alpha P + \beta C^T \mathbf{1}
%&= \alpha \mat{1 \\ 0 \\ 1 \\ 1 \\ 0} + \beta
%\mat{
%	1 & 1 & 0 & 0 \\
%	0 & 0 & 0 & 0 \\
%	0 & 0 & 1 & 0 \\
%	0 & 0 & 0 & 1 \\
%	0 & 0 & 0 & 0 \\
%} \mat{1 \\ 1 \\ 1 \\ 1}
%\\
%& = \alpha \mat{1 \\ 0 \\ 1 \\ 1 \\ 0} + \beta \mat{2 \\ 0 \\ 1 \\ 1 \\ 0} \\
%& = \mat{\alpha + 2 \beta \\ 0 \\ \alpha + \beta \\ \alpha + \beta \\ 0}
%\end{split}
%\end{equation*}
%%
The objective function is to minimize the total cost, $\min(\sum M)$.
There are some constraints. The resources consumed by each user, $R_u$, cannot 
exceed those provided by each router.
%
$$ RC \le k P^T $$
%
Each user should be connected to one router.
%
$$ C \bf{1} = \bf{1} $$
%
The final optimization model is:
%
\begin{align}
\text{Minimize} \quad & \sum M \nonumber \\
\text{Subject to}	\quad & M = \alpha P + \beta C^T \bf{1} \\
& RC \le k P^T \\
& C \bf{1} = \bf{1}
\end{align}

\paragraph*{Implementation}
Lets build a python program using pulp to solve this model. First we notice that 
the data in {\tt examA.dat} is written in a unusual format, that cannot be 
parsed by python easily. There is a script provided in {\tt AMMMLabHeuristics 
v1.1.zip} called {\tt DATParser.py} which reads {\tt .dat} files into a 
dictionary. Hopefully the script is licensed using the GPLv3 so there is a 
chance to reuse for this ocassion.

The script is based on python 2, but can be updated to the 3 using the handy 
tool {\tt 2to3}. After some adjustments, a {\tt .json} file can be generated 
based on the {\tt .dat} files.
%
\begin{textcode}
% python dat2json.py data/*.dat
% cat data/examA.json | fold -s
{"c": 20, "accessPoints": 5, "k": 7, "posUsers": [[1, 1], [2, 3], [0, 1], [4, 
1], [1, 2], [2, 2], [0, 1], [1, 1], [3, 4], [2, 4]], "f": 100, 
"posAccessPoints": [[2, 3], [1, 2], [1, 1], [0, 2], [1, 3]], "d": 3, "bigM": 
100000, "cr": [2, 3, 4, 1, 2, 2, 1, 2, 3, 4], "userLocs": 10}
\end{textcode}
%
Now the data can be loaded in python using a json decoder. First, some constants 
should be loaded:

\begin{pycode}
data = json.load(sys.stdin)

ALPHA = data['f']
BETA = data['c']
K = data['k']
R = data['cr']
N_AP = data['accessPoints']

U = range(0, len(R))
A = range(0, N_AP)
\end{pycode}
%
Then, the variables are defined.
%
\begin{pycode}
# The vector P, where a 1 states a router in that position
P = vector_var(A, 'P', 0, 1, LpBinary)

# The matrix C, so C_ua is 1 if the user u is connected to the router at a.
C = matrix_var(U, A, 'C', 0, 1, LpBinary)

# The cost of the router at the access point a
M = vector_var(A, "M", 0, kind=LpContinuous)
\end{pycode}
%
To continue with the constraints.
%
\begin{pycode}
# The problem
prob = LpProblem(problem, LpMinimize)

# Constraint 1
for a in A:
	prob += M[a] == ALPHA * P[a] + BETA * lpSum([C[u][a] for u in U])

# Constraint 2
for a in A:
	prob += lpSum([R[u] * C[u][a] for u in U]) <= K * P[a]

# Constraint 3
for u in U:
	prob += lpSum([C[u][a] for a in A]) == 1

# Objective function
prob += lpSum([M[a] for a in A])
\end{pycode}
%
Finally, after the execution an optimal solution is found, with an amount of 600 
for the minimization function.
%
\begin{textcode}
% python 1a.py < data/examA.json
Solver status: Optimal
Cost: sum(M) = 600.0
Matrix C:
  0  1  0  0  0
  0  0  0  1  0
  0  0  1  0  0
  1  0  0  0  0
  0  0  1  0  0
  0  1  0  0  0
  1  0  0  0  0
  1  0  0  0  0
  0  1  0  0  0
  0  0  0  1  0
Vector M:
  160.0  160.0  140.0  140.0  0.0
Vector P:
  1  1  1  1  0
Vector R:
  2  3  4  1  2  2  1  2  3  4
Router at 0 used 4/7 by users [3, 6, 7]
Router at 1 used 7/7 by users [0, 5, 8]
Router at 2 used 6/7 by users [2, 4]
Router at 3 used 7/7 by users [1, 9]
\end{textcode}

\subsection*{B}
\textsl{Assume that we have one additional requirement: routers can only be 
connected to users that are at distance at most $d$. To implement this 
constraint we know, for each user $u \in U$, its location $(u_x, u_y)$ and for 
each access point $a \in A$, its location $(a_x, a_y)$. Write down the 
mathematical formulation and extend the OPL model. Save this file as 
\texttt{modelB.mod} and send it through ``El Racó''. Use \texttt{modelA.mod} 
again as problem data.}

\paragraph*{Model}

To add the notion of distance, lets define the distance between a user $u$ and 
the access point $a$ as the matrix $Z \in \mathbb{R}^{U \times A}$ defined as
%
$$ Z_{ua} = \sqrt{(a_x - u_x)^2 + (a_y - u_y)^2} $$
%
To restrict the distance less than $d$, a new contraint should be added
%
$$ C_u^T Z_u \le d $$
%
Leading the new model
%
\begin{align}
\text{Minimize} \quad & \sum M \nonumber\\
\text{Subject to}	 \setcounter{equation}{0} \quad & M = \alpha P + \beta C^T 
\bf{1} \\
& RC \le k P^T \\
& C \bf{1} = \bf{1} \\
& C_u^T Z_u \le d
\end{align}

\paragraph*{Implementation}

The matrix $Z$ is first precomputed based on the users positions $V$, and the 
access point positions $W$.
%
\begin{pycode}
def dist(a, b):
	return math.sqrt((a[0] - b[0])**2 + (a[1] - b[1])**2)

# Precompute a table with all the possible distances
Z = []
for u in U:
	Zu = []
	for a in A:
		Zu.append(dist(V[u], W[a]))
	Z.append(Zu)
\end{pycode}
%
The new constraint is added to the model:
%
\begin{pycode}
# Constraint 4
for u in U:
	prob += lpSum([C[u][a] * Z[u][a] for a in A]) <= D
\end{pycode}
%
After the execution, the new results have the same cost, but a different layout.
%
\begin{textcode}
% python 1b.py < data/examA.json
Solver status: Optimal
Cost: sum(M) = 600.0
Matrix C:
  0  1  0  0  0
  1  0  0  0  0
  0  0  1  0  0
  0  0  1  0  0
  0  0  0  0  1
  0  0  0  0  1
  0  0  0  0  1
  0  0  0  0  1
  0  1  0  0  0
  1  0  0  0  0
Vector M:
  140.0  140.0  140.0  0.0  180.0
Vector P:
  1  1  1  0  1
Vector R:
  2  3  4  1  2  2  1  2  3  4
Matrix Z:
  2.24  1.00  0.00  1.41  2.00
  0.00  1.41  2.24  2.24  1.00
  2.83  1.41  1.00  1.00  2.24
  2.83  3.16  3.00  4.12  3.61
  1.41  0.00  1.00  1.00  1.00
  1.00  1.00  1.41  2.00  1.41
  2.83  1.41  1.00  1.00  2.24
  2.24  1.00  0.00  1.41  2.00
  1.41  2.83  3.61  3.61  2.24
  1.00  2.24  3.16  2.83  1.41
Router at 0 used 7/7 by users [1, 9] at distances [0.00, 1.00]
Router at 1 used 5/7 by users [0, 8] at distances [1.00, 2.83]
Router at 2 used 5/7 by users [2, 3] at distances [1.00, 3.00]
Router at 4 used 7/7 by users [4, 5, 6, 7] at distances [1.00, 1.41, 2.24, 2.00]
\end{textcode}


\end{document}
